
\documentclass[11pt]{charter}
\usepackage{adjustbox}

% El títulos de la memoria, se usa en la carátula y se puede usar el cualquier lugar del documento con el comando \ttitle
\titulo{Sistema de control y monitoreo de riego } 

% Nombre del posgrado, se usa en la carátula y se puede usar el cualquier lugar del documento con el comando \degreename
\posgrado{Carrera de Especialización en Sistemas Embebidos} 
%\posgrado{Carrera de Especialización en Internet de las Cosas} 
%\posgrado{Carrera de Especialización en Intelegencia Artificial}
%\posgrado{Maestría en Sistemas Embebidos} 
%\posgrado{Maestría en Internet de las cosas}

% Tu nombre, se puede usar el cualquier lugar del documento con el comando \authorname
\autor{Ariel Alejandro Cerfoglia} 

% El nombre del director y co-director, se puede usar el cualquier lugar del documento con el comando \supname y \cosupname y \pertesupname y \pertecosupname
\director{Esp. Ing. Nelson Ariel Fortunatti}
\pertenenciaDirector{ITBA} 
% FIXME:NO IMPLEMENTADO EL CODIRECTOR ni su pertenencia
%\codirector{} % si queda vacio no se deberíá incluir 
%\pertenenciaCoDirector{}

% Nombre del cliente, quien va a aprobar los resultados del proyecto, se puede usar con el comando \clientename y \empclientename
\cliente{Prof. Mg. Ing. Osvaldo P. Ivani}
\empresaCliente{Smartium S.A}

% Nombre y pertenencia de los jurados, se pueden usar el cualquier lugar del documento con el comando \jurunoname, \jurdosname y \jurtresname y \perteunoname, \pertedosname y \pertetresname.
%\juradoUno{Nombre y Apellido (1)}
%\pertenenciaJurUno{pertenencia (1)} 
%\juradoDos{Nombre y Apellido (2)}
%\pertenenciaJurDos{pertenencia (2)}
%\juradoTres{Nombre y Apellido (3)}
%\pertenenciaJurTres{pertenencia (3)}
 
\fechaINICIO{23 de octubre de 2020}		%Fecha de inicio de la cursada de GdP \fechaInicioName
\fechaFINALPlanificacion{11 de diciembre de 2020} 	%Fecha de final de cursada de GdP
\fechaFINALTrabajo{19 de Octubre de 2021}		%Fecha de defensa pública del trabajo final


\begin{document}

\maketitle
\thispagestyle{empty}
\pagebreak


\thispagestyle{empty}
{\setlength{\parskip}{0pt}
\tableofcontents{}
}
\pagebreak


\section{Registros de cambios}
\label{sec:registro}


\begin{table}[ht]
\label{tab:registro}
\centering
\begin{tabularx}{\linewidth}{@{}|c|X|c|@{}}
\hline
\rowcolor[HTML]{C0C0C0} 
Revisión & \multicolumn{1}{c|}{\cellcolor[HTML]{C0C0C0}Detalles de los cambios realizados} & Fecha      \\ \hline
1.0      & Creación del documento                                          & 23/10/2020 \\ \hline
1.1      & Incorporación de incisos 1 a 6                                  & 03/11/2020 \\ \hline
1.2      & Incorporación de historias de usuario \newline
		   Incorporación de activity on node \newline 									   
		   Corrección de incisos 1 a 6                           & 11/11/2020 \\ \hline
1.3		 & Modificación de diagrama de activity on node\newline
		   Incorporación de incisos 8 a 13						 & 23/11/2020  \\ \hline
1.4		 & Incorporación de incisos 14 a 17 					 & 28/11/2020 \\ \hline  
\end{tabularx}
\end{table}

\pagebreak



\section{Acta de constitución del proyecto}
\label{sec:acta}

\begin{flushright}
Buenos Aires, \fechaInicioName
\end{flushright}

\vspace{2cm}

Por medio de la presente se acuerda con el Ing. \authorname\hspace{1px} que su Trabajo Final de la \degreename\hspace{1px} se titulará ``\ttitle'', consistirá esencialmente en el prototipo de un sistema para el control y monitoreo de riego en campos de producción agropecuaria, y tendrá un presupuesto preliminar estimado de 600 hs de trabajo y 400.000 pesos, con fecha de inicio \fechaInicioName\hspace{1px} y fecha de presentación pública \fechaFinalName.

Se adjunta a esta acta la planificación inicial.

\vfill

% Esta parte se construye sola con la información que hayan cargado en el preámbulo del documento y no debe modificarla
\begin{table}[ht]
\centering
\begin{tabular}{ccc}
\begin{tabular}[c]{@{}c@{}}Ariel Lutenberg \\ Director posgrado FIUBA\end{tabular} & \hspace{2cm} & \begin{tabular}[c]{@{}c@{}}\clientename \\ \empclientename \end{tabular} \vspace{2.5cm} \\ 
\multicolumn{3}{c}{\begin{tabular}[c]{@{}c@{}} \supname \\ Director del Trabajo Final\end{tabular}} \vspace{2.5cm} \\
%\begin{tabular}[c]{@{}c@{}}\jurunoname \\ Jurado del Trabajo Final\end{tabular}     &  & \begin{tabular}[c]{@{}c@{}}\jurdosname\\ Jurado del Trabajo Final\end{tabular}  \vspace{2.5cm}  \\
%\multicolumn{3}{c}{\begin{tabular}[c]{@{}c@{}} \jurtresname\\ Jurado del Trabajo Final\end{tabular}} \vspace{.5cm}                                                                     
\end{tabular}
\end{table}




\section{Descripción técnica-conceptual del proyecto a realizar}
\label{sec:descripcion}

\begin{consigna}{black}
%El objetivo es que el lector en una o dos páginas entienda de qué se trata el proyecto y cuáles son sus desafíos, su motivación y su importancia.
%Se debe destacar claramente cuál es el valor que agrega el proyecto a realizar. ``El presente proyecto se destaca especialmente por incorporar tal cosa... Esto lo diferencia de otros sistemas similares en que ...''

%Puede ser útil incluir en esta sección la respuesta a alguna de estas preguntas:

%\begin{itemize}
%\item ¿Cómo se vincula este proyecto con la misión de la organización?
%\item ¿Cómo se inserta este proyecto en el modelo de negocio de la organización?
%\item ¿Ayuda a la explicación si se incluye un lienzo Canvas del Modelo de Negocio?
%\item ¿En qué estado del ciclo de vida está el producto que se desea reemplazar o mejorar?
%\item ¿Cuales son las necesidades que debe satisfacer?
%\item ¿Por dónde pasa la innovación?
%\end{itemize}

%La descripción técnica-conceptual \textbf{debe incluir al menos un diagrama en bloques del sistema }y una frase como la siguiente: ``En la Figura \ref{fig:diagBloques} se presenta el diagrama en bloques del sistema. Se observa que...''. Luego recién más abajo de haber puesto esta frase se pone la figura. La regla es que las figuras nunca pueden ir antes de ser mencionadas en el texto, porque sino el lector no entiende por qué de pronto aparece una figura.

La revolución 4.0 es una realidad cada vez más cercana. El mercado del internet de las cosas crece a ritmos acelerados, permitiendo la conectividad de cada vez más dispositivos. El sector agropecuario no ha sido ajeno a estos cambios, y en los recientes años han surgido un creciente número de empresas agtech, vinculadas a la tecnología para la agricultura.

Este proyecto pretende realizar un sistema de control y monitoreo de riego que sea incorporado a un sistema de monitoreo preexistente de una empresa agtech. Se ampliará su funcionalidad y brindará al productor agropecuario un mejor manejo del agua en sus cultivos. 

El sistema se comunicará de forma inalámbrica con un servidor en la empresa \empclientename . Podrá recibir  y transmitir datos desde y hacia el mismo, respectivamente. El protocolo utilizado en dicha comunicación será el protocolo LoRa, caracterizado por su eficiencia para transmitir información a largas distancias y con baja potencia; resulta especialmente útil en zonas sin cobertura celular. En la figura \ref{fig:redLoRa} se muestra la estructura básica de la red LoRa.

\begin{figure}[htpb]
\centering 
\includegraphics[width=.7\textwidth]{./Figuras/redLora.png}
\caption{Estructura de una red LoRa}
\label{fig:redLoRa}
\end{figure}

El control del riego se realizará por medio de sensores y actuadores que permitirán un funcionamiento autónomo del sistema. Los actuadores consistirán en siete electroválvulas y una bomba de riego trifásica que serán accionados de manera indirecta por sus correspondientes relés. El sensado se realizará con hasta 8 sensores de presión o caudal, los cuales permitirán conocer el estado del circuito hidráulico.

Una pantalla táctil desplegará la información importante del sistema y su configuración de forma local. La gestión de las reglas de control de riego se realizará desde un servidor por un técnico o personal especializado. Esto permitirá al operario utilizar el sistema de un modo intuitivo y con un rápido acceso a la información importante.

Por último, un subsistema dedicado al diagnóstico y gestión de fallas, alertará de manera local y remota en caso de errores en el funcionamiento.

Se espera con este proyecto brindar un prototipo de un producto confiable, intuitivo y que mejore la gestión del agua por parte de los productores agropecuarios. En la figura \ref{fig:diagBloques} se presenta un diagrama del sistema completo.


\vspace{25px}

\begin{figure}[htpb]
\centering 
\includegraphics[width=0.9\textwidth]{./Figuras/DiagramaBloques.png}
\caption{Diagrama en bloques del sistema}

\label{fig:diagBloques}
\end{figure}

\vspace{25px}

%El tamaño de la tipografía en la figura debe ser adecuado para que NO pase lo que ocurre acá, donde el lector debe esforzarse para poder leer el texto. Los colores usados en el diagrama deben ser adecuados, tal que ayuden a comprender mejor el diagrama.
\end{consigna}


\section{Identificación y análisis de los interesados}
\label{sec:interesados}

\begin{consigna}{black} 
%Nota: (borrar esto y todas las consignas en color rojo antes de entregar este documento).
 
%Es inusual que una misma persona esté en más de un rol, incluso en proyectos chicos.
 
%Si se considera que una persona cumple dos o más roles, entonces sólo dejarla en el rol más importante. Por ejemplo:

%\begin{itemize}
%\item Si una persona es Cliente pero también colabora u orienta, dejarla solo como Cliente.
%\item Si una persona es el Responsable, no debe ser colocado también como Miembro del equipo.
%\end{itemize}

%Pero en cambio sí es usual que el Cliente y el Auspiciante sean el mismo, por ejemplo.

\begin{table}[ht]
%\caption{Identificación de los interesados}
%\label{tab:interesados}
\begin{tabularx}{\linewidth}{|p{3cm}|p{5.5cm}|X|l|@{}}
\hline
\rowcolor[HTML]{C0C0C0} 
Rol           & Nombre y Apellido & Organización 	& Puesto 	\\ \hline
Auspiciante, cliente e impulsor.   & \clientename      &\empclientename	& Director de Ingeniería\\ \hline
Responsable   & \authorname       & FIUBA        	& Alumno 	\\ \hline
Colaboradores & -                 & -             	& -       	\\ \hline
Orientador    & \supname	      & \pertesupname 	& Director	Trabajo final \\ \hline
Equipo        & -   		      & -             	& -       	\\ \hline
Opositores    & -                 & -             	& -       	\\ \hline
Usuario final & Productores agropecuarios & -       & -       	\\ \hline
\end{tabularx}
\end{table}

%El Director suele ser uno de los Orientadores.

%No dejar celdas vacías; si no hay nada que poner en una celda colocar un signo ``-''.

%No dejar filas vacías; si no hay nada que poner en una fila entonces eliminarla.

%Sería deseable listar a continuación de la tabla las principales características de cada interesado.
 
%Por ejemplo:
\begin{itemize}
\item Cliente: El cliente dispone de poco tiempo. 
%\item Equipo: Juan Perez, suele pedir licencia porque tiene un familiar con una enfermedad. Planificar considerando esto.
%\item Orientador: María Gómez, nos va a poder ayudar mucho con la gestión de impuestos.
\end{itemize}

\end{consigna}

\clearpage

\section{1. Propósito del proyecto}
\label{sec:proposito}
El propósito de este proyecto es el prototipado de un sistema de control y monitoreo de riego. El objetivo del prototipo es la creación de un producto que permita extender las prestaciones  del servicio de monitoreo que brinda la empresa \empclientename.

%\begin{consigna}{black}
%¿Por qué se hace el proyecto? ¿Qué se quiere lograr? 

%Se recomienda que sea solo un párrafo que empiece diciendo ``El propósito de este proyecto es...''.
%\end{consigna}


\section{2. Alcance del proyecto}
\label{sec:alcance}
%\begin{consigna}{red}
%¿Qué se incluye y que no se incluye en este proyecto?
%Se refiere al trabajo a hacer para entregar el producto o resultado especificado. 
%Explicitar todo lo quede comprendido dentro del alcance del proyecto.
%Explicitar además todo lo que no quede incluido (``El presente proyecto no incluye...'')
%\end{consigna}

El proyecto incluirá:
\begin{itemize}
 \item[$-$] Control automático de riego.
\item[$-$] Sistema de detección de fallas.
\item[$-$] Comunicación con servidor desde red LoRa.
\item[$-$] Sistema de alimentación secundario.
\item[$-$] Interfaz gráfica.
\item[$-$] Soporte para control y configuración desde un servidor remoto.
\item[$-$] Página web en servidor con funcionalidad mínima para control y visualización de datos importantes.
\end{itemize}


El presente proyecto no incluye:
\begin{itemize}
\item[$-$] Sensores remotos.
\item[$-$] Pagina web completa con capacidad para gestionar todas las configuraciones disponibles.
\end{itemize}

\section{3. Supuestos del proyecto}
\label{sec:supuestos}
%\begin{consigna}{red}
%``Para el desarrollo del presente proyecto se supone que: ...''

%\begin{itemize}
%\item Supuesto 1
%\item Supuesto 2...
%\end{itemize}

%Por ejemplo, se podrían incluir supuestos respecto a disponibilidad de tiempo y recursos humanos y materiales, sobre la factibilidad técnica de distintos aspectos del proyecto, sobre otras cuestiones que sean necesarias para el éxito del proyecto como condiciones macroeconómicas o reglamentarias.
%\end{consigna}

Para el desarrollo del presente proyecto se supone que:
\begin{itemize}
\item[$-$] El proyecto puede ser desarrollado en tiempo y forma en base a un análisis previo.
\item[$-$] El cliente proveerá el hardware necesario para realizar el desarrollo.
\end{itemize}

\clearpage

\section{4. Requerimientos}
\label{sec:requerimientos}

%\begin{consigna}{red}
%Los requerimientos deben numerarse y de ser posible agruparlos por afinidad:
%\begin{enumerate}
%\item Grupo de requerimientos asociados con...
%	\begin{enumerate}
%	\item Requerimiento 1
%	\item Requerimiento 2
%	\item Requerimiento 3 (prioridad menor)
%	\end{enumerate}
%\item Grupo de requerimientos asociados con...
%	\begin{enumerate}
%	\item Requerimiento 1
%	\item Requerimiento 2 (prioridad menor)
%	\end{enumerate}
%\end{enumerate}

%Leyendo los requerimientos se debe poder interpretar cómo será el proyecto y su funcionalidad.

%De ser posible indicar cómo se obtuvieron cada uno de los requerimientos 

%Indicar claramente cuál es la prioridad entre los distintos requerimientos. 

%No olvidarse de que los requerimientos incluyen a las regulaciones y normas vigentes!!!

%Y al escribirlos seguir las siguientes reglas:
%\begin{itemize}
%\item Ser breve y conciso (nadie lee cosas largas). 
%\item Ser específico: no dejar lugar a confusiones.
%\item Expresar los requerimientos en términos que sean cuantificables y medibles.
%\end{itemize}
%\end{consigna}

\begin{enumerate}
	\item \textbf{\textit{Hardware:}}
	\begin{enumerate}
		\item El sistema debe poder accionar desde 1 (una) hasta 8 (ocho) 			  	electroválvulas.
		\item El sistema debe ser capaz de activar y desactivar el contactor de una bomba de riego trifásica.
		\item Se deben poder manejar entre 1 (uno) y 7 (siete) sensores de presión o caudal.
		\item El sistema debe tener conexión con una red LoRaWAN.
		\item En caso de interrupción del suministro eléctrico, el sistema deberá funcionar por hasta 2 semanas solo con alertas sonoras y envío de notificaciones al servidor.
	\end{enumerate}		
	
	%\vspace{25px}
	\item \textbf{\textit{Firmware:}}
	\begin{enumerate}
		\item Debe poder recibir y transmitir datos desde y hacia una base de datos.
		\item Debe poder configurar rutinas de riego. Cada rutina de riego tendrá una fecha de inicio y una fecha de finalización.
		\item Debe detectar la falta de conectividad con la red LoRaWAN.
		\item Debe ejecutar acciones solicitadas desde un servidor. 
	\end{enumerate}
	
	\item \textbf{\textit{Interfaz gráfica:}}
	\begin{enumerate}
		\item Debe ser aprobada por el cliente.
		\item Debe ser intuitiva y fácil de entender.		
		\item Debe mostrar los datos de los sensores de presión y caudal.
		\item Debe mostrar el estado de las electroválvulas. 
		\item Debe mostrar el estado de la bomba de riego.
	\end{enumerate}

	%\item \textbf{\textit{No funcionales:}}
	
\end{enumerate}

\section{Historias de usuarios (\textit{Product backlog})}
\label{sec:backlog}

%\begin{consigna}{red}
%Descripción: En esta sección se deben incluir las historias de usuarios y su ponderación (\textit{history points}). Recordar que las historias de usuarios son descripciones cortas y simples de una característica contada desde la perspectiva de la persona que desea la nueva capacidad, generalmente un usuario o cliente del sistema. La ponderación es un número entero que representa el tamaño de la historia comparada con otras historias de similar tipo.
%\end{consigna}
\begin{enumerate}

	\item Como ingeniero agrónomo, quiero controlar remotamente el riego para aumentar la calidad en los cultivos.
	\begin{table}[h]
		\begin{center}
			\begin{tabular}{|c|c|c|c|} \hline
				\multicolumn{4}{|c|}{\textbf{\textit{History points}}} \\\hline
  			Conocimiento & Complejidad & Volumen & Incertidumbre \\\hline
    			  1 	   &     13       &   13     &      8        \\\hline
			\end{tabular}
		\end{center}
	\end{table}


	 \item Como ingeniero agrónomo, quiero gestionar rutinas de riego para organizar las plantaciones.
	 \begin{table}[h]
		\begin{center}
			\begin{tabular}{|c|c|c|c|} \hline
				\multicolumn{4}{|c|}{\textbf{\textit{History points}}} \\\hline
  			Conocimiento & Complejidad & Volumen & Incertidumbre \\\hline
    			  3 	   &     8       &   8     &      3        \\\hline
			\end{tabular}
		\end{center}
	\end{table}

\clearpage

	\item Como ingeniero agrónomo. quiero estar al tanto de los problemas para reducir riesgos.
	\begin{table}[h]
		\begin{center}
			\begin{tabular}{|c|c|c|c|} \hline
				\multicolumn{4}{|c|}{\textbf{\textit{History points}}} \\\hline
  			Conocimiento & Complejidad & Volumen & Incertidumbre \\\hline
    			  5 	   &     21       &   21     &      21        \\\hline
			\end{tabular}
		\end{center}
	\end{table}


	\item Como operario, quiero conocer la información importante de un modo sencillo para dar una rápida respuesta al supervisor.
	\begin{table}[h]
		\begin{center}
			\begin{tabular}{|c|c|c|c|} \hline
				\multicolumn{4}{|c|}{\textbf{\textit{History points}}} \\\hline
  			Conocimiento & Complejidad & Volumen & Incertidumbre \\\hline
    			  13	   &     3       &   3     &      13        \\\hline
			\end{tabular}
		\end{center}
	\end{table}

	
	\item Como dueño del campo, quiero consultar de un modo simple el estado del riego para tomar decisiones estratégicas.
	\begin{table}[h]
		\begin{center}
			\begin{tabular}{|c|c|c|c|} \hline
				\multicolumn{4}{|c|}{\textbf{\textit{History points}}} \\\hline
  			Conocimiento & Complejidad & Volumen & Incertidumbre \\\hline
    			  2 	   &     2       &   5     &      5        \\\hline
			\end{tabular}
		\end{center}
	\end{table}

	
	\item Como ingeniero agrónomo, quiero que el sistema me informe de fallas en el suministro eléctrico para replanificar el riego.
	\begin{table}[H]
		\begin{center}
			\begin{tabular}{|c|c|c|c|} \hline
				\multicolumn{4}{|c|}{\textbf{\textit{History points}}} \\\hline
  			Conocimiento & Complejidad & Volumen & Incertidumbre \\\hline
    			  21 	   &     1       &   1     &      1        \\\hline
			\end{tabular}
		\end{center}
	\end{table}

	

	\item Como operario, quiero poder regar en caso de perdida de conexión con el servidor para terminar con las tareas del día.
	\begin{table}[H]
		\begin{center}
			\begin{tabular}{|c|c|c|c|} \hline
				\multicolumn{4}{|c|}{\textbf{\textit{History points}}} \\\hline
  			Conocimiento & Complejidad & Volumen & Incertidumbre \\\hline
    			  8 	   &     5       &   2     &      2        \\\hline
			\end{tabular}
		\end{center}
	\end{table}


\end{enumerate}


\section{5. Entregables principales del proyecto}
\label{sec:entregables}

%\begin{consigna}{red}
%Cosas como: 
\begin{itemize}
%\item Manual de uso
\item[$-$] Diagrama esquemático.
\item[$-$] Código fuente.
\item[$-$] Video demostrativo.
%\item Diagrama de instalación
\item[$-$] Informe final.

\end{itemize}

%\end{consigna}


\section{6. Desglose del trabajo en tareas}
\label{sec:wbs}

%\begin{consigna}{red}
%Se recomienda mostrar el WBS mediante una lista indexada:

%\begin{enumerate}
%\item Grupo de tareas 1
%	\begin{enumerate}
%	\item Tarea 1 (tantas hs)
%	\item Tarea 2 (tantas hs)
%	\item Tarea 3 (tantas hs)
%	\end{enumerate}
%\item Grupo de tareas 2
%	\begin{enumerate}
%	\item Tarea 1 (tantas hs)
%	\item Tarea 2 (tantas hs)
%	\item Tarea 3 (tantas hs)
%	\end{enumerate}
%	\item Grupo de tareas 3
%	\begin{enumerate}
%	\item Tarea 1 (tantas hs)
%	\item Tarea 2 (tantas hs)
%	\item Tarea 3 (tantas hs)
%	\item Tarea 4 (tantas hs)
%	\item Tarea 5 (tantas hs)
%	\end{enumerate}
%\end{enumerate}

%Cantidad total de horas: (tantas hs)

%Se recomienda que no haya ninguna tarea que lleve más de 40 hs. 
%\end{consigna}

\begin{enumerate}
	\item \textbf{ \textit{Planificación:}} 50hs
	\begin{enumerate}
		\item Redacción de planificación: 30hs 
		\item Análisis y validación con cliente : 5hs
		\item Diseño de arquitectura y validación: 15hs
	\end{enumerate}
	
	 \item \textbf{\textit{Firmware:}} 322hs
	 \begin{enumerate}
	 	\item Asimilación de herramientas a utilizar: 30hs
	 	\item Estructura base: 60hs
	 	\begin{enumerate}
	 		\item Driver de sensores de presión: 15hs
	 		\item Driver de caudalímetros: 15hs
	 		\item Tareas base: 30hs
	 	\end{enumerate}
	 	\item Gestor de rutinas de riego: 30hs
	 	\item Integración a Firmware 0.1v: 2hs
	 	\item Interfaz Gráfica: 40hs
	 	\item Integración a Firmware 0.2v: 2hs
	 	\item Decodificador de mensajes externos[básico]: 30hs
	 	\item Integración a Firmware 0.3v: 2hs
	 	\item Monitor de fallas y driver LoRaWAN: 30hs
	 	\item Integración a Firmware 0.4v: 2hs
	 	\item Gestor de reglas de control: 40hs
	 	\item Integración a Firmware 1.0v: 4hs
	 	\item Decodificador de mensajes externo[Completo]: 50hs
	 \end{enumerate}
	 
	 \item \textbf{\textit{Hardware:}} 140hs
	 \begin{enumerate}
	 	\item Diseño de sistema de alimentación primaria: 10hs
	 	\item Diseño de sistema de alimentación secundaria: 30hs
	 	\item Análisis de proveedores: 20hs
	 	\item Desarrollo de placa completa: 80hs
	 	\begin{enumerate}
	 		\item Diseño de Esquematico: 40hs
	 		\item Diseño de PCB: 40hs
	 	\end{enumerate}
	 \end{enumerate}
	 	
	 \item \textbf{\textit{Web:}} 60hs
	 \begin{enumerate}
	 	\item Investigación y asimilación de herramientas: 30hs
	 	\item Diseño de interfaz web: 30hs
	 \end{enumerate}
	 
	 \item \textbf{\textit{Redacción de informe final:}} 60hs
\end{enumerate}

Cantidad total de horas: 632hs

\clearpage

\section{7. Diagrama de Activity On Node}
\label{sec:AoN}

%\begin{consigna}{red}
%Armar el AoN a partir del WBS definido en la etapa anterior. 
%La figura \ref{fig:AoN} fue elaborada con el paquete latex tikz y pueden consultar la siguiente referencia \textit{online}:
%\url{https://www.overleaf.com/learn/latex/LaTeX_Graphics_using_TikZ:_A_Tutorial_for_Beginners_(Part_3)\%E2\%80\%94Creating_Flowcharts}
%\end{consigna}


\begin{figure}[htpb]
\centering 
\includegraphics[width=1.1\textwidth]{./Figuras/activityOnNode2.png}
\caption{Diagrama en \textit{Activity on Node}}
\label{fig:AoN}
\end{figure}


%Indicar claramente en qué unidades están expresados los tiempos.
%De ser necesario indicar los caminos semicríticos y analizar sus tiempos mediante un cuadro.
%Es recomendable usar colores y un cuadro indicativo describiendo qué representa cada color, como se muestra en el siguiente ejemplo:

\clearpage


\section{8. Diagrama de Gantt}
\label{sec:gantt}

%\begin{consigna}{red}
%Utilizar el software Gantter for Google Drive o alguno similar para dibujar el diagrama de Gantt.

%Existen muchos programas y recursos \textit{online} para hacer diagramas de gantt, entre las cuales destacamos:

%\begin{itemize}
%\item Planner
%\item GanttProject
%\item Trello + \textit{plugins}. En el siguiente link hay un tutorial oficial: \\ %\url{https://blog.trello.com/es/diagrama-de-gantt-de-un-proyecto}
%\item Creately, herramienta online colaborativa. \\\url{https://creately.com/diagram/example/ieb3p3ml/LaTeX}
%\item Se puede hacer en latex con el paquete \textit{pgfgantt}\\ \url{http://ctan.dcc.uchile.cl/graphics/pgf/contrib/pgfgantt/pgfgantt.pdf}
%\end{itemize}

%Pegar acá una captura de pantalla del diagrama de Gantt, cuidando que la letra sea suficientemente grande como para ser legible. 
%Si el diagrama queda demasiado ancho, se puede pegar primero la ``tabla'' del Gantt y luego pegar la parte del diagrama de barras del diagrama de Gantt.

%Configurar el software para que en la parte de la tabla muestre los códigos del EDT (WBS).\\
%Configurar el software para que al lado de cada barra muestre el nombre de cada tarea.\\
%Revisar que la fecha de finalización coincida con lo indicado en el Acta Constitutiva.

%En la figura \ref{fig:gantt}, se muestra un ejemplo de diagrama de gantt realizado con el paquete de \textit{pgfgantt}. En la plantilla pueden ver el código que lo genera y usarlo de base para construir el propio.

%\begin{figure}[htbp]
%\begin{center}
%\begin{ganttchart}{1}{12}
%  \gantttitle{2020}{12} \\
%  \gantttitlelist{1,...,12}{1} \\
%  \ganttgroup{Group 1}{1}{7} \\
%  \ganttbar{Task 1}{1}{2} \\
%  \ganttlinkedbar{Task 2}{3}{7} \ganttnewline
%  \ganttmilestone{Milestone o hito}{7} \ganttnewline
%  \ganttbar{Final Task}{8}{12}
%  \ganttlink{elem2}{elem3}
%  \ganttlink{elem3}{elem4}
%\end{ganttchart}
%\end{center}
%\caption{Diagrama de gantt de ejemplo}
%\label{fig:gantt}
%\end{figure}

%\end{consigna}

\begin{figure}[htpb]
\centering 
\includegraphics[width=\textwidth]{./Figuras/gantT.png}
\caption{Tabla de tareas en Diagrama de Gantt}
\label{fig:redLoRa}
\end{figure}

\begin{figure}[htpb]
\centering
\includegraphics[width=0.85\textwidth]{./Figuras/gantG.png}
\caption{Diagrama de Gantt}
\label{fig:redLoRa}
\end{figure}

\section{9. Matriz de uso de recursos de materiales}
\label{sec:recursos}


\begin{table}[H]
\label{tab:recursos}
\centering
\begin{tabularx}{\linewidth}{@{}|c|X|p{0.6cm}|p{3.2cm}|X|X|@{}} \hline
\cellcolor[HTML]{C0C0C0} & \cellcolor[HTML]{C0C0C0} & \multicolumn{4}{c|}{\cellcolor[HTML]{C0C0C0}Recursos requeridos (horas)} \\ \cline{3-6} 
\multirow{-2}{*}{\cellcolor[HTML]{C0C0C0}\begin{tabular}[c]{@{}c@{}}Código\\ WBS\end{tabular}} & \multirow{-2}{*}{\cellcolor[HTML]{C0C0C0}\begin{tabular}[c]{@{}c@{}}Nombre \\ tarea\end{tabular}} & PC & Placa de desarrollo & Módulo LoRaWAN & Plataforma Smartium \\ \hline
1 & Planificación  & 50 &  &  &  \\ \hline
2.1 & Asimilar\newline herramientas & 30 &  &  &  \\ \hline
2.2 & Estructura\newline base  & 60 & 20 &  &  \\ \hline
2.3 & Gestor de\newline rutinas & 30 & 5 &  &  \\ \hline
2.5 & Interfaz\newline gráfica & 40 & 40 &  &  \\ \hline
2.7 & Decodificador\newline básico & 30 & 30 &  & 10 \\ \hline
2.9 & Monitor y\newline driver\newline LoRaWAN & 30 & 30 & 20 & 10 \\ \hline
2.11 & Gestor de\newline reglas & 40 & 40 & 20 & 10 \\ \hline 
2.13 & Decodificador\newline completo & 50 & 50 & 40 & 40 \\ \hline
3 & Hardware & 140 &  &  &  \\ \hline
4 & Web & 60 & 30 & 30 & 60 \\ \hline

\end{tabularx}%
\end{table}


\section{10. Presupuesto detallado del proyecto}
\label{sec:presupuesto}

%\begin{consigna}{red}
%Si el proyecto es complejo entonces separarlo en partes:
%\begin{itemize}
%\item Un total global, indicando el subtotal acumulado por cada una de las áreas.
%%\item El desglose detallado del subtotal de cada una de las áreas.
%\end{itemize}

%IMPORTANTE: No olvidarse de considerar los COSTOS INDIRECTOS.

%\end{consigna}

\begin{table}[htpb]
\centering
\begin{tabularx}{\linewidth}{@{}|X|c|r|r|@{}}
\hline
\rowcolor[HTML]{C0C0C0} 
\multicolumn{4}{|c|}{\cellcolor[HTML]{C0C0C0}COSTOS DIRECTOS} \\ \hline
\rowcolor[HTML]{C0C0C0} 
Descripción &
  \multicolumn{1}{c|}{\cellcolor[HTML]{C0C0C0}Cantidad} &
  \multicolumn{1}{c|}{\cellcolor[HTML]{C0C0C0}Valor unitario} &
  \multicolumn{1}{c|}{\cellcolor[HTML]{C0C0C0}Valor total} \\ \hline
 Hora de Ingeniero Jr. & 
  \multicolumn{1}{c|}{632} &
  \multicolumn{1}{c|}{354,44} &
  \multicolumn{1}{r|}{224006} \\ \hline
 Kit de desarrollo &  
  \multicolumn{1}{c|}{1} &
  \multicolumn{1}{c|}{20000} &
  \multicolumn{1}{r|}{20000} \\ \hline
\multicolumn{1}{|l|}{Módulo LoRaWAN} &
   1 &
\multicolumn{1}{c|}{6000}  & 
\multicolumn{1}{r|}{6000}  \\ \hline
\multicolumn{3}{|c|}{SUBTOTAL} &
  \multicolumn{1}{r|}{\textbf{\textit{250006}}} \\ \hline
\rowcolor[HTML]{C0C0C0} 
\multicolumn{4}{|c|}{\cellcolor[HTML]{C0C0C0}COSTOS INDIRECTOS} \\ \hline
\rowcolor[HTML]{C0C0C0} 
Descripción &
  \multicolumn{1}{c|}{\cellcolor[HTML]{C0C0C0}Cantidad} &
  \multicolumn{1}{c|}{\cellcolor[HTML]{C0C0C0}Valor unitario} &
  \multicolumn{1}{c|}{\cellcolor[HTML]{C0C0C0}Valor total} \\ \hline
\multicolumn{1}{|l|}{Costos indirectos (\% 60 de los costos directos)} &
\multicolumn{1}{c|}{1}   &
\multicolumn{1}{c|}{150000} &
\multicolumn{1}{r|}{150000} \\ \hline

\multicolumn{3}{|c|}{SUBTOTAL} &
  \multicolumn{1}{r|}{\textbf{\textit{150000}}} \\ \hline
\rowcolor[HTML]{C0C0C0}
\multicolumn{3}{|c|}{TOTAL} & \multicolumn{1}{r|}{\textit{\textbf{400006}}}
   \\ \hline
\end{tabularx}%
\end{table}

\clearpage

\section{11. Matriz de asignación de responsabilidades}
\label{sec:responsabilidades}
%\begin{consigna}{red}
%Establecer la matriz de asignación de responsabilidades y el manejo de la autoridad completando la siguiente tabla:

\begin{table}[htpb]
\raggedleft
\begin{adjustbox}{max width=15cm}
\begin{tabular}{|p{1cm}|p{3.2cm}|c|c|c|}
\hline
\rowcolor[HTML]{C0C0C0} 
\cellcolor[HTML]{C0C0C0} &
  \cellcolor[HTML]{C0C0C0} &
  \multicolumn{3}{c|}{\cellcolor[HTML]{C0C0C0}Listar todos los nombres y roles del proyecto} \\ \cline{3-5} 
\rowcolor[HTML]{C0C0C0} 
\cellcolor[HTML]{C0C0C0} &
  \cellcolor[HTML]{C0C0C0} &
  Responsable &
  Orientador &
  Cliente \\ \cline{3-5} 
\rowcolor[HTML]{C0C0C0} 
\multirow{-3}{*}{\cellcolor[HTML]{C0C0C0}\begin{tabular}[c]{@{}c@{}}Código\\ WBS\end{tabular}} &
  \multirow{-3}{*}{\cellcolor[HTML]{C0C0C0}Nombre de la tarea} &
  \authorname &
  \supname &
  \clientename \\ \hline
1.1 & Redacción de\newline planificación  & P  & C & I \\ \hline
1.2 & Análisis y\newline validación\newline con cliente & P & I & A \\ \hline
1.3 & Diseño de\newline arquitectura y\newline validación & P & I & A \\ \hline
2.1 & Asimilación de\newline herramientas\newline a utilizar &  P  & I & I \\ \hline
2.2 & Estructura base & P & I & A \\ \hline
2.3 & Gestor de rutinas de riego & P & I & I \\ \hline
2.4 & Integración\newline a Firmware 0.1v & P & I & A \\ \hline
2.5 & Interfaz Gráfica & P &  I & I \\ \hline
2.6 & Integración\newline a Firmware 0.2v & P & I  & A \\ \hline
2.7 & Decodificador\newline de mensajes\newline externos [básico] & P & I & I \\ \hline
2.8 & Integración\newline a Firmware 0.3v & P & I  & A \\ \hline
2.9 & Monitor de fallas y driver LoRaWAN & P  & I & I \\ \hline
2.10 & Integración\newline a Firmware 0.4v  & P & I  & A \\ \hline
2.11 & Gestor de reglas de control & P & I  & I \\ \hline
2.12 & Integración\newline a Firmware 1.0v & P & I & A \\ \hline
2.13 & Decodificador\newline de mensajes\newline externos[completo] & P & I  & A \\ \hline
3.1 & Diseño de\newline sistema de \newline alimentación\newline primaria & P & I  & I \\ \hline
3.2 & Diseño de \newline sistema de\newline alimentación\newline secundaria & P & I  & I  \\ \hline
3.3 & Análisis de\newline proveedores & P & I & C \\ \hline
3.4 & Desarrollo de\newline placa completa & P & I & A \\ \hline
4.1 & Investigación\newline y asimilación de\newline herramientas & P & I  & I \\ \hline
4.2 & Diseño de\newline interfaz web & P  & I  & A \\ \hline
5 & Redacción de\newline informe final & P  & C  & I \\ \hline
\end{tabular}%
\end{adjustbox}

\end{table}

{\footnotesize
Referencias:
\begin{itemize}
	\item P = Responsabilidad Primaria
	\item S = Responsabilidad Secundaria
	\item A = Aprobación
	\item I = Informado
	\item C = Consultado
\end{itemize}
} %footnotesize

%Una de las columnas debe ser para el Director, ya que se supone que participará en el proyecto.
%A su vez se debe cuidar que no queden muchas tareas seguidas sin ``A'' o ``I''.

%Importante: es redundante poner ``I/A'' o ``I/C'', porque para aprobarlo o responder consultas primero la persona debe ser informada.

%\end{consigna}

\section{12. Gestión de riesgos}
\label{sec:riesgos}

%\begin{consigna}{red}
%a) Identificación de los riesgos (al menos cinco) y estimación de sus consecuencias:
 
%Riesgo 1: detallar el riesgo (riesgo es algo que si ocurre altera los planes previstos)
%\begin{itemize}
%\item Severidad (S): mientras más severo, más alto es el número (usar números del 1 al 10).\\
%Justificar el motivo por el cual se asigna determinado número de severidad (S).
%\item Probabilidad de ocurrencia (O): mientras más probable, más alto es el número (usar del 1 al 10).\\
%Justificar el motivo por el cual se asigna determinado número de (O). 
%\end{itemize}   

%Riesgo 2:
%\begin{itemize}
%\item Severidad (S): 
%\item Ocurrencia (O):
%\end{itemize}

%Riesgo 3:
%\begin{itemize}
%\item Severidad (S): 
%\item Ocurrencia (O):
%\end{itemize}

\begin{itemize}
 \item[•] \textbf{Riesgo 1: Retraso en el proyecto}
  \begin{itemize}
  	\item[$-$] \textit{\textbf{Severidad (S): 6}} - El retraso en el proyecto implica la no finalización de la especialización. El retraso en el proyecto no tiene un gran impacto sobre el cliente por ser este un prototipo sin fecha de venta establecida.
  	\item[$-$] \textit{\textbf{Probabilidad de Ocurrencia (O): 8}} - El retraso en el proyecto final es algo muy común debido a la longitud del mismo.
  \end{itemize}
\end{itemize}

\begin{itemize}
 \item[•] \textbf{Riesgo 2: El microcontolador no posee la potencia necesaria}
  \begin{itemize}
  	\item[$-$] \textit{\textbf{Severidad (S): 9}} - Frente a la imposibilidad de completar la implementación por falta de recursos, el proyecto en su totalidad se verá comprometido.
  	\item[$-$] \textit{\textbf{Probabilidad de Ocurrencia (O): 5}} - La necesidad de incorporar una interfaz gráfica genera una carga sustancial sobre el sistema embebido y las consideraciones en tamaño de memoria y potencia de procesamiento pueden estar subdimensionadads. 
  \end{itemize}
\end{itemize}

\begin{itemize}
 \item[•] \textbf{Riesgo 3: Programa para desarrollo de interfaz gráfica presenta restricciones limitantes.}
  \begin{itemize}
  	\item[$-$] \textit{\textbf{Severidad (S): 4}} - La interfaz gráfica presenta varias implementaciones posibles, lo cual, minimiza el riesgo de encontrar limitaciones en el diseño.
  	\item[$-$] \textit{\textbf{Probabilidad de Ocurrencia (O): 4}} - El desarrollo de interfaces gráficas en sistemas embebidos es un requerimiento muy común. Los programas para desarrollo de interfaces gráficas en sistemas embebidos son bastante completos. 
  \end{itemize}
\end{itemize}

\begin{itemize}
 \item[•] \textbf{Riesgo 4: Daño en el kit de desarrollo}
  \begin{itemize}
  	\item[$-$] \textit{\textbf{Severidad (S): 10}} - El kit de desarrollo es un elemento de costo elevado. Todo el proyecto será construido con el kit, por lo cual, su destrucción comprometerá la consecución del proyecto.
  	\item[$-$] \textit{\textbf{Probabilidad de Ocurrencia (O): 2}} - Quemar el kit de desarrollo es algo improbable si se es cuidadoso en su manipulación.
  \end{itemize}
\end{itemize}

\begin{itemize}
 \item[•] \textbf{Riesgo 5: Pérdida de los archivos fuente del firmware}
  \begin{itemize}
  	\item[$-$] \textit{\textbf{Severidad (S): 10}} - El firmware constituye la parte más importante del proyecto. Su pérdida puede reprensentar un retraso significativo.
  	\item[$-$] \textit{\textbf{Probabilidad de Ocurrencia (O): 1}} - Se utilizará un control de versiones para la construcción del firmware. 
  \end{itemize}
\end{itemize}

\clearpage
%b) Tabla de gestión de riesgos:      (El RPN se calcula como RPN=SxO)

\begin{table}[htpb]
\centering
\begin{tabularx}{\linewidth}{@{}|X|c|c|c|c|c|c|@{}}
\hline
\rowcolor[HTML]{C0C0C0} 
Riesgo & S & O & RPN & S* & O* & RPN* \\ \hline
1. Retraso en el proyecto & 6 & 8 & \textcolor{red}{48} &  6  & \textbf{\textcolor{green}{4}}  & \textbf{\textcolor{green}{24}}     \\ \hline
2. El microcontolador no posee la potencia necesaria       & 9 & 5 & \textcolor{red}{45} & \textbf{\textcolor{green}{5}} & 5  & \textbf{\textcolor{green}{25}}  \\ \hline
3. Programa para desarrollo de interfaz gráfica presenta restricciones limitantes      & 4 & 4 & \textbf{\textcolor{green}{16}}   &    &    &      \\ \hline
4. Daño en el kit de desarrollo  & 10  & 2 &  \textbf{\textcolor{green}{20}}   &    &    &      \\ \hline
5. Pérdida de los archivos fuente del firmware & 10  & 1 & \textbf{\textcolor{green}{10}} &    &    &      \\ \hline
\end{tabularx}%
\end{table}

%Criterio adoptado: 
Se tomarán medidas de mitigación en los riesgos cuyos números de RPN sean mayores a 30

%Nota: los valores marcados con (*) en la tabla corresponden luego de haber aplicado la mitigación.

%c) Plan de mitigación de los riesgos que originalmente excedían el RPN máximo %establecido:
 
\begin{itemize}
 \item[•] \textbf{Riesgo 1: plan de mitigación}
  \begin{itemize}
  	\item[$-$] \textit{\textbf{Probabilidad de Ocurrencia (O*): 4}} - Se destinarán más horas de las planificadas para mitigar la probabilidad de haber subdimensionado el proyecto.
  \end{itemize}
\end{itemize}

\begin{itemize}
 \item[•] \textbf{Riesgo 2: plan de mitigación}
  \begin{itemize}
  	\item[$-$] \textit{\textbf{Severidad (S*): 5}} - Se diseñará el sistema con una HAL (hardware abstraction layer).
  \end{itemize}
\end{itemize}
%  Nueva asignación de S y O, con su respectiva justificación:
%  - Severidad (S): mientras más severo, más alto es el número (usar números del 1 al %10).
%          Justificar el motivo por el cual se asigna determinado número de severidad %(S).
%  - Probabilidad de ocurrencia (O): mientras más probable, más alto es el número (usar del 1 al 10).
%          Justificar el motivo por el cual se asigna determinado número de (O).

%Riesgo 2: plan de mitigación (si por el RPN fuera necesario elaborar un plan de mitigación).
 
%Riesgo 3: plan de mitigación (si por el RPN fuera necesario elaborar un plan de mitigación).

%\end{consigna}


\section{13. Gestión de la calidad}
\label{sec:calidad}

%\begin{consigna}{red}
%Para cada uno de los requerimientos del proyecto indique:
%\begin{itemize} 
%\item Req \#1: copiar acá el requerimiento.

\textit{\textbf{Verificación y validación de requerimientos:}}
\begin{itemize}
	\item \textit{\textbf{1.1 Hardware:}} El sistema debe poder accionar de 1 (una) hasta 8 (ocho) electroválvulas.
	\begin{itemize}
		
		\item \textit{ \textbf{Verificación}}
		\begin{itemize}
			\item El sistema cambia cada una de las 8 salidas al activarse la zona correspondiente.
		\end{itemize}
		
		\item \textit{\textbf{Validación}}
		\begin{itemize}
			\item El sistema acciona los relés necesarios para la activación de las correspondientes electroválvulas. 			
		\end{itemize}
	\end{itemize}
	
	\item \textit{\textbf{1.2 Hardware:}} El sistema debe ser capaz de activar y desactivar el contactor de una bomba de riego trifásica.
	\begin{itemize}
		\item \textit{ \textbf{Verificación}}
		\begin{itemize}
			\item El sistema activa y desactiva la salida de la bomba cuando se decide activar y desactivar una zona, respectivamente.
		\end{itemize}
		
		\item \textit{\textbf{Validación}}
		\begin{itemize}
			\item El sistema activa y desactiva el relé que accionará el contactor de la bomba.
		\end{itemize}
	\end{itemize}

\clearpage
	\item \textit{\textbf{1.3 Hardware:}} Se deben poder manejar entre 1 (uno) y 7(siete) sensores de presión o caudal.
	\begin{itemize}
		\item \textit{ \textbf{Verificación}} 
		\begin{itemize}
			\item El sistema activa el modo falla en caso de detectar más de 2 atmósferas entre los sensores de presión.
			\item El sistema activa el modo falla en caso de detectar más de 1 metro cúbico por segundo.
		\end{itemize}

		\item \textit{\textbf{Validación}}
		\begin{itemize}
			\item El sistema reconoce adecuadamente el protocolo de comunicación de los sensores.
		\end{itemize}
	\end{itemize}
	
	\item \textit{\textbf{1.4 Hardware:}} El sistema debe tener conexión con una red LoRaWAN
	\begin{itemize}
		\item \textit{ \textbf{Verificación}}
		\begin{itemize}
			\item El módulo LoRaWAN recibe los mensajes correctos por parte del microcontrolador.
		\end{itemize}
		
		\item \textit{\textbf{Validación}}
		\begin{itemize}
			\item Se realiza una comunicación efectiva con la plataforma Smartium.
		\end{itemize}
	\end{itemize}
	
	\item \textit{\textbf{1.5 Hardware:}} En caso de interrupción del suministró eléctrico, el sistema deberá funcionar por hasta 2 semanas solo con alertas sonoras y envío de notificaciones al servidor.
	\begin{itemize}
		\item \textit{ \textbf{Verificación}} 
		\begin{itemize}
			\item Se calcula que la energía de la batería permita 2 semanas de uso para transmisión de mensajes y alertas sonoras.
		\end{itemize}
		
		\item \textit{\textbf{Validación}}
		\begin{itemize}
			\item Se deja funcionando el sistema durante 2 semanas sin suministro eléctrico.
		\end{itemize}
	\end{itemize}

	\item \textit{\textbf{2.1 Firmware:}} Debe poder recibir y trasmitir datos desde y hacia una base de datos
	\begin{itemize}
		\item \textit{ \textbf{Verificación}}
		\begin{itemize}
			\item El sistema envía mensajes al módulo LoRaWAN con las peticiónes necesarias para actuar sobre la base de datos.
			\item El sistema puede procesar mensajes para recibir datos desde la base de datos.
		\end{itemize}
		
		\item \textit{\textbf{Validación}}
		\begin{itemize}
			\item El sistema logra colocar datos en la base de datos.
			\item El sistema logra recibir datos de la base de datos.
		\end{itemize}
	\end{itemize}
	
	\item \textit{\textbf{2.2 Firmware:}} Debe poder configurar rutinas de riego. Cada rutina de riego tendrá una fecha de inicio y una fecha de finalización.
	\begin{itemize}
		\item \textit{ \textbf{Verificación}}
		\begin{itemize}
			\item El sistema configura correctamente las variables internas para fijar una fecha de inicio y una fecha de finalización.
		\end{itemize}
		
		\item \textit{\textbf{Validación}}
		\begin{itemize}
			\item El sistema activa y desactiva las salidas en base a una rutina de riego fijada.
		\end{itemize}
	\end{itemize}

\clearpage
	
	\item \textit{\textbf{2.3 Firmware:}} Debe detectar la falta de conectividad con la red LoRaWAN.
	\begin{itemize}
		\item \textit{ \textbf{Verificación}}
		\begin{itemize}
			\item El sistema entra correctamente al estado "sin conectividad".
		\end{itemize}
		
		\item \textit{\textbf{Validación}}
		\begin{itemize}
			\item Al desconectar la antena del módulo LoRaWAN el sistema entra al estado de "sin conectividad". Desde la interfaz gráfica lo informa al usuario.
		\end{itemize}					
	\end{itemize}
	
	\item \textit{\textbf{2.4 Firmware:}} Debe ejecutar acciones solicitadas desde un servidor.
	\begin{itemize}
		\item \textit{ \textbf{Verificación}}
		\begin{itemize}
			\item Se suministran al sistema los comandos que enviaría un servidor y este es capaz de reconocerlos.
		\end{itemize}
		
		\item \textit{\textbf{Validación}}
		\begin{itemize}
			\item Se solicita una acción desde el servidor y el sistema la realiza.
		\end{itemize}
	\end{itemize}
	
	\item \textit{\textbf{3.1 Interfaz gráfica:}} Debe ser aprobada por el cliente
	\begin{itemize}
		\item \textit{ \textbf{Verificación:}} N/A
		
		\item \textit{\textbf{Validación}}
		\begin{itemize}
			\item Es mostrada al cliente. Este la aprueba.	
		\end{itemize}			
	\end{itemize}
	
	\item \textit{\textbf{3.2 Interfaz gráfica:}} Debe ser intuitiva y facil de entender
	\begin{itemize}
		\item \textit{ \textbf{Verificación:}} N/A
		
		\item \textit{\textbf{Validación}}
		\begin{itemize}
			\item Es presentada al cliente. Este la testea y la aprueba.
		\end{itemize}
	\end{itemize}
	
	\item \textit{\textbf{3.3 Interfaz gráfica:}} Debe mostrar los datos de los sensores de presión y caudal.
	\begin{itemize}
		\item \textit{ \textbf{Verificación:}} N/A
		
		\item \textit{\textbf{Validación}}
		\begin{itemize}
			\item El cliente lo aprueba.
		\end{itemize}
	\end{itemize}
	
	\item \textit{\textbf{3.4 Interfaz gráfica:}} Debe mostrar el estado de las electroválvulas.
	\begin{itemize}
		\item \textit{ \textbf{Verificación:}} N/A
		
		\item \textit{\textbf{Validación}}
		\begin{itemize}
			\item El cliente lo aprueba.
		\end{itemize}
	\end{itemize}
	
	\item \textit{\textbf{3.5 Interfaz gráfica:}} Debe mostrar el estado de la bomba de riego.
	\begin{itemize}
		\item \textit{ \textbf{Verificación:}} N/A
		
		\item \textit{\textbf{Validación}}
		\begin{itemize}
			\item El cliente lo aprueba.
		\end{itemize}
	\end{itemize}
	
\end{itemize}					
	
	
	


%\begin{itemize}
%\item Verificación para confirmar si se cumplió con lo requerido antes de mostrar el sistema al cliente. Detallar 
%\item Validación con el cliente para confirmar que está de acuerdo en que se cumplió con lo requerido. Detallar  
%\end{itemize}

%\end{itemize}

%Tener en cuenta que en este contexto se pueden mencionar simulaciones, cálculos, revisión de hojas de datos, consulta con expertos, mediciones, etc.

%\end{consigna}

\clearpage

\section{14. Comunicación del proyecto}
\label{sec:comunicaciones}

El plan de comunicación del proyecto es el siguiente:

\begin{table}[htpb]
\centering
\begin{tabularx}{\linewidth}{@{}|C{3cm}|C{1.7cm}|C{2.1cm}|C{1.6cm}|C{2.9cm}|C{2cm}|@{}}
\hline
\rowcolor[HTML]{C0C0C0} 
\multicolumn{6}{|c|}{\cellcolor[HTML]{C0C0C0}PLAN DE COMUNICACIÓN DEL PROYECTO}           \\ \hline
\rowcolor[HTML]{C0C0C0} 
¿Qué comunicar? & Audiencia & Propósito & Frecuencia & Método de\newline comunicacion & Responsable \\ \hline
  Plan de proyecto          &    Cliente y Director       &    Presentar planificación    &  Única vez    &   Email   &  \authorname           \\ \hline
   Avance     &  Cliente y Director    &   Mantener informado & Semanal  &   Email    &   \authorname           \\ \hline
  Hitos  &   Cliente     &   Mantener informado   &  N/A          & Videoconferencia &             
  \authorname  \\ \hline
  Informe de avance & Jurados y Director & Ser evaluado &  Única vez & Email   &                
  \authorname 	\\ \hline
  Informe final     & Jurados y Director & Obtener titulación & Única vez & Exposición Oral  & \authorname            \\ \hline
\end{tabularx}
\end{table}

\section{15. Gestión de compras}
\label{sec:compras}

N/A

\clearpage

\section{16. Seguimiento y control}
\label{sec:seguimiento}

%\begin{consigna}{red}
%Para cada tarea del proyecto establecer la frecuencia y los indicadores con los se seguirá su avance y quién será el responsable de hacer dicho seguimiento y a quién debe comunicarse la situación (en concordancia con el Plan de Comunicación del proyecto).

%El indicador de avance tiene que ser algo medible, mejor incluso si se puede medir en \% de avance. Por ejemplo,se pueden indicar en esta columna cosas como ``cantidad de conexiones ruteadeas'' o ``cantidad de funciones implementadas'', pero no algo genérico y ambiguo como ``\%'', porque el lector no sabe porcentaje de qué cosa.

%\end{consigna}

%\begin{longtable}{|m{1cm}|m{3.5cm}|m{2.2cm}|m{2cm}|m{3cm}|m{1.5cm}|}
%\hline
%\rowcolor[HTML]{C0C0C0} 
%\multicolumn{6}{|c|}{\cellcolor[HTML]{C0C0C0}SEGUIMIENTO DE AVANCE}                                                                       \\ \hline
%\rowcolor[HTML]{C0C0C0} 
%Tarea del WBS 			& Indicador de avance & Frecuencia de reporte & Resp. de seguimiento & Persona a ser informada & Método de comunic. \\ \hline
%\endfirsthead

%\hline
%\rowcolor[HTML]{C0C0C0} 
%\multicolumn{6}{c}{\cellcolor[HTML]{C0C0C0}SEGUIMIENTO DE AVANCE}                                                                       \\ \hline
%\rowcolor[HTML]{C0C0C0} 
%Tarea del WBS 			& Indicador de avance & Frecuencia de reporte & Resp. de seguimiento & Persona a ser informada & Método de comunic. \\ \hline
%\endhead

%\multicolumn{6}{c}{Continúa}
%\endfoot

%\endlastfoot

%1.1	& Fecha de inicio  & Única vez al comienzo & \authorname & \clientename, \supname & email \\ \hline
%2.1	& Avance de las subtareas  & Mensual mientras dure la tarea & \authorname & \clientename, \supname & email \\ \hline

%\end{longtable}
%\multicolumn{2}{|c|}{\textcolor{white}{Tarea del WBS}}
%\begin{longtable}[!htpb]
%\begin{tabularx}{\linewidth}{@{}|X|X|X|X|X|X|@{}}
\begin{longtable}{@{}|r | p{2.7cm}|C{1.9cm}|C{1.9cm}|C{2cm}|C{2.4cm}|C{2cm}|@{}}
\hline
\rowcolor[HTML]{C0C0C0} 
\multicolumn{7}{|c|}{\cellcolor[HTML]{C0C0C0}SEGUIMIENTO DE AVANCE}                                                                       \\ \hline
\rowcolor[HTML]{C0C0C0} 
\multicolumn{2}{|c|}{Tarea del WBS} & Indicador de avance & Frecuencia de reporte & Resp. de seguimiento & Persona a ser informada & Método de comunic. \\ \hline

 \textbf{\textcolor{pink}{1.}} & Planificación & Incisos  & Semanal & Dr. Ariel Lutenberg & Mg. Osvaldo Ivani & Formulario  \\ \hline
 \textbf{\textcolor{cyan}{2.1.}} & Asimilar\newline herramientas & Horas dedicadas & Semanal & Esp. Nelson Fortunatti & Mg. Osvaldo Ivani & Informe\newline semanal \\ \hline
 \textbf{\textcolor{cyan}{2.2.}} & Estructura base & Funciones & Semanal & Esp. Nelson Fortunatti & Mg. Osvaldo Ivani & Informe\newline semanal \\ \hline
 \textbf{\textcolor{cyan}{2.3.}} & Gestor de\newline rutinas & Funciones & Semanal & Esp. Nelson Fortunatti & Mg. Osvaldo Ivani & Informe\newline semanal \\ \hline
 \textbf{\textcolor{cyan}{2.4.}} & Firmware 0.2v   & Fallas & Semanal & Esp. Nelson Fortunatti &   Mg. Osvaldo Ivani & Conferencia \\ \hline
 \textbf{\textcolor{cyan}{2.5.}} & Interfaz Gráfica  & Pantallas & Semanal & Esp. Nelson Fortunatti & Mg. Osvaldo Ivani & Informe\newline semanal \\ \hline
 \textbf{\textcolor{cyan}{2.6.}} & Firmware 0.3v  & Fallas & Semanal & Esp. Nelson Fortunatti &  Mg. Osvaldo Ivani & Conferencia \\ \hline
 \textbf{\textcolor{cyan}{2.7.}} & Decodificador básico & Funciones & Semanal & Esp. Nelson Fortunatti & Mg. Osvaldo Ivani & Informe\newline semanal \\ \hline
 \textbf{\textcolor{cyan}{2.8.}} & Firmware 0.3v & Fallas & Semanal & Esp. Nelson Fortunatti & Mg. Osvaldo Ivani & Conferencia \\ \hline
 \textbf{\textcolor{cyan}{2.9.}} & Monitor y\newline Driver  & Funciones & Semanal & Esp. Nelson Fortunatti & Mg. Osvaldo Ivani & Informe\newline semanal \\ \hline
 \textbf{\textcolor{cyan}{2.10.}} & Firmware 0.4v  & Fallas & Semanal & Esp. Nelson Fortunatti &  Mg. Osvaldo Ivani & Conferencia \\ \hline
 \textbf{\textcolor{cyan}{2.11.}} & Gestor de reglas & Funciones  & Semanal & Esp. Nelson Fortunatti & Mg. Osvaldo Ivani & Informe\newline semanal \\ \hline
 \textbf{\textcolor{cyan}{2.12.}} & Firmware 1.0v & Fallas  & Semanal & Esp. Nelson Fortunatti &  Mg. Osvaldo Ivani & Conferencia \\ \hline
 \textbf{\textcolor{cyan}{2.13.}} & Decodificador completo  & Horas & Semanal &  Esp. Nelson Fortunatti & Mg. Osvaldo Ivani & Informe\newline semanal \\ \hline
 \textbf{\textcolor{green}{3.1.}} & Alimentación primaria  & Horas & Semanal & Esp. Nelson Fortunatti & Mg. Osvaldo Ivani & Informe\newline semanal \\ \hline
 \textbf{\textcolor{green}{3.2.}} & Alimentación secundaria & Horas & Semanal & Esp. Nelson Fortunatti & Mg. Osvaldo Ivani & Informe\newline semanal \\ \hline
 \textbf{\textcolor{green}{3.3.}} & Analisis de\newline proveedores  & Horas & Semanal & Esp. Nelson Fortunatti & Mg. Osvaldo Ivani & Informe\newline semanal \\ \hline
 \textcolor{green}{3.4.} & Placa completa  & Módulos & Semanal & Esp. Nelson Fortunatti & Mg. Osvaldo Ivani & Conferencia \\ \hline
 \textbf{\textcolor{orange}{4.1.}} & Investigar y\newline asimilar\newline herramientas  &  Horas  & Semanal & Esp. Nelson Fortunatti & Mg. Osvaldo Ivani & Informe\newline semanal \\ \hline
 \textbf{\textcolor{orange}{4.2.}} & Diseño web & Horas & Semanal & Esp. Nelson Fortunatti & Mg. Osvaldo Ivani & Informe\newline semanal \\ \hline
 \textbf{\textcolor{red}{5.}} & Redacción\newline de informe & Páginas & Al finalizar  & Esp. Nelson Fortunatti & Mg. Osvaldo Ivani & Email \\ \hline


%\end{tabularx}%
%}
\end{longtable}


\begin{tabular}{|c|c|c|c|c|} \hline
	\cellcolor{pink} Planificación & \cellcolor{cyan} Firmware & \cellcolor{green} Hardware & \cellcolor{orange} Web & \cellcolor{red} Informe Final\\ \hline
\end{tabular}

\section{17. Procesos de cierre}    
\label{sec:cierre}

%\begin{consigna}{red}
%Establecer las pautas de trabajo para realizar una reunión final de evaluación del proyecto, tal que contemple las siguientes actividades:

%\begin{itemize}
%\item Pautas de trabajo que se seguirán para analizar si se respetó el Plan de Proyecto original:
% - Indicar quién se ocupará de hacer esto y cuál será el procedimiento a aplicar. 
%\item Identificación de las técnicas y procedimientos útiles e inútiles que se utilizaron, y los problemas que surgieron y cómo se solucionaron:
% - Indicar quién se ocupará de hacer esto y cuál será el procedimiento para dejar registro.
%\item Indicar quién organizará el acto de agradecimiento a todos los interesados, y en especial al equipo de trabajo y colaboradores:
%  - Indicar esto y quién financiará los gastos correspondientes.
%\end{itemize}
%\end{consigna}

\begin{itemize}
	\item Será contrastada la duración real de cada tarea con la duración planteada. La fecha de finalización de cada tarea será documentada en una planilla de cálculo.
	\item Se realizará un Diagrama de Gantt que refleje los tiempos reales en el desarrollo del proyecto.
	\item Se realizará un análisis en retrospectiva para extraer conclusiones.
	\item Se invitará a todos los interesados a una exposición oral, y en especial, a los implicados en el proyecto. Se agradecerá formalmente a todos aquellos que contribuyeron en el desarrollo del proyecto. 
\end{itemize}

\end{document}
